\documentclass[letterpaper]{article}
\usepackage{aaai}
\usepackage{times}
\usepackage{helvet}
\usepackage{courier}
\usepackage{graphicx}
\usepackage{graphicx}

\begin{document}

\title{CS 440: Project 1\\ A Sudoku Puzzle Solving Agent \\ Due
  March 1}
\author{Scott Wallace}

\maketitle

\begin{abstract}
This handout describes Project 1: A Sudoku Puzzle Solving Agent. The
project will be executed in a team environment. Students will
implement at least two solution strategies for the problem and then
evaluate their strategies with a set of experiments. A written report
will document both the implementation and the experimental evaluation.
\end{abstract}

\section{Overview}
This project is to be carried out in a 3-4 person teams.  The goals
are threefold: first, to further develop your AI/Search toolkit;
second, to help develop your technical writing skills; and third, to
give you some research savvy (this will come in handy regardless of
whether you decide to head off to graduate school).

For this project, you will implement a Sudoku puzzle-solving agent using a
set of strategies that you choose. At a minimum, you must implement
at least two strategies from amongst the approaches covered in class.
Strategies need not be completely different, but they should differ in
more than a few lines of code (e.g., switching from depth-first to
breadth-first alone, won't cut it). 

If you are a firm believer that
only one strategy from class is reasonable in this context, or if you
wish to expand your report, you may also consider multiple {\em
  implementations} of a particular strategy.  In this case again, you
should ensure that your implementations differ in more than a few
lines of code from one another.  You can do this by changing data
structures, or heuristics (and possibly other things as well). Be sure
to document your changes so you can adequately report them.

The project has three basic parts: solution design/implementation,
evaluation and write-up.



\section{Sudoku}

% this is a comment
% the [h] in {figure}[h] indicates I want the figure to be 'here'
% its also possible to add other location preferences [tbh] 
% indicates top, bottom, or here, are all ok.  Note that these
% are only preferences, LaTeX will make its own judgement ;)
\begin{figure}[h] 
\begin{center}
\includegraphics[width=3in]{sudoku.pdf}
\caption{A 9x9 Sudoku Puzzle}
\label{fig:sudoku}  % a label can be added after a caption and referenced later
\end{center}
\end{figure}

A standard Sudoku puzzle consists of a 9x9 grid which is sub-divided
into 9 3x3 squares as show in 
% Here I'm referencing the figure, the ~ will generate a space but ensure that 
% the line doesn't break there.
Figure~\ref{fig:sudoku}. 
Each of the 81 cells contains either a number in the range 1-9, or is empty. The
puzzle is solved by filling in each of the blanks with number 1-9
subject to the following constraints: each row must contain all the
numbers 1-9, each column must contain all the numbers 1-9, and each of
the 9 3x3 squares must contain the numbers 1-9.

Sudoku puzzles can be generalized for any $n^2$x$n^2$ grid.  The
standard puzzle described above is the $n=3$ puzzle.

\section{Solution Design and Implementation}

You can implement your sudoku solving agent using a programming
language of your choice. I suggest Python, but, if your teammates are
willing, you may want to use another language all together.

Your implementation must consist of a program that is executable via
the command line on one of the lab machine's operating systems. The
program should accept sudoku puzzles from {\tt stdin} and print
results to {\tt stdout}. Different strategies should be selected using
a flag sent to the program. Flags should be single characters starting
with {\em a}, in alphabetic order. Running the program with no flags
should print a help message describing the strategy associated with
each flag.

% $ enters and exits 'inline' math mode
A puzzle is represented as a $n^4$ string of characters that begins
with the digit in the upper-left corner of the puzzle and proceeds
from left to right and top to bottom. Thus the standard 9x9 puzzle
contains $3^{4}=81$ characters. Each character is either a number 1-9
or a `.' to represent an empty cell.  Larger puzzles (e.g.,
$n=4$ or $n=5$) can be described using alpha-numeric characters for
digits. As with standard hexadecimal notation, assume $a=10$ and
character values increase through $z$ and then continue with upper
case letters $A \dots Z$. This provides enough ``digits'' to generate
an $n=7$ puzzle. Your program should handle $n=3$ puzzles.
If you are able to solve larger puzzles, extra credit may be granted to your work.

To solve the Sudoku puzzles, you should select at least two strategies
(you may do more; ambitious projects
will earn extra credit). Each strategy should be implemented to solve
puzzles as specified above. Be sure to document your code so we can
take a look at your implementations. A good project will select
strategies in a thoughtful manner, either to illustrate a point, or to
explore particular performance issues.

\section{Evaluation}

In Computer Science, research papers typically revolve around a
specific problem that is solved using a method proposed and
implemented by the paper's authors. The authors must then demonstrate
why their approach is interesting enough to warrant a paper. This is
to say, they must evaluate their work in some way that is meaningful
to the readers.

For this project, you must similarly provide an evaluation of your
solution strategies for Sudoku.  There is no single recipe for what
constitutes a ``good'' evaluation; much will depend on exactly what
strategies you choose to implement. The overall goal of the evaluation
should be to characterize the strengths and weaknesses of each
approach within the Sudoku puzzle domain.

A reasonable way to perform evaluations for this environment is with a
set of experiments each of which ends with a set of data that can be
used to generate a graph or table. All projects should contain at least
two graphs/tables, and each graph/table should illustrate
a different property of your solutions. Put another way, each graph or table should help to
answer a question that someone else might have about your work.  For example, most
users of a Sudoku problem-solving agent would probably be interested
to know how quickly the agent can solve puzzles using its built-in set
of strategies.  Users may care about average time and also about the worst
case time. Likewise, if you swap out a data structure, you should
consider what theoretical impact that might have and then try to
produce a graph to illustrate the difference in
practice. \footnote{I'm willing to entertain other approaches for
  evaluating your work, but
  please run them by me first.}.

 At a minimum, you should be sure to include two graphs illustrating
 different properties of your solutions. One graph/table should
illustrate the amount of CPU time required to solve a fixed set of
benchmark problems provided on the website.  Ambitious projects may
go beyond evaluating the solution strategies and also try to
characterize the space of Sudoku puzzles (that is, the space beyond the small sample
set that is provided).

\section{Write Up}

To complete the project you must submit a write up that documents your
approach along with the executable code.

The writeup should be 4-5 pages in AAAI two-column format.  This is
the standard format used in publications and will equal roughly 8
pages in microsoft word's default format.  It is also the format that
this document is written in.

The report should contain
an {\em abstract} that briefly summarizes the paper in a single
paragraph. Next, you should write a section with the title {\em
  introduction} that lays out the problem. Following this you may
choose to write one section per strategy and evaluate each strategy
within that section, or to write a single {\em implementation} section
describing all of your strategies (possibly using subsections to
signpost the strategies) and then a subsequent section entitled {\em
  evaluation} to describe the experiments performed.  In either case,
these middle sections should constitute the bulk of the writing and
describe your implementations in enough detail for me to understand
what you've done and think through the ramifications {\em without}
needing to look at your code.  Your paper should end with a section
entitled {\em conclusions and future work} in which you should outline
the ``take away message'' from your experiments as well as a set of
future enhancements/implementations or experiments that would be
interesting to pursue if you had unlimited time.

I strongly suggest that you do your writing in LaTeX.  It's a bit of a
drag to learn, but I'll provide some samples that will get you going.
It's also much less likely that you'll lose work due to your text
editor choking on a figure than if you use one of those ``new-fangled''
WYSIWYG word processors.

\section{Turning it in}

You need to turn in three things for this assignment:

% the enumerate environment creates a numbered list
% other list environments include:
% \begin{itemize}
% \item blah blah
% \end{itemize}
%
% and 
% \begin{description}
% \item[Item 1] is great
% \item[Item 2] is better
% \end{description}
\begin{enumerate}
\item The write up for your implementation and experiments as
  described above.

\item A executable file (or python source code) that can be run on the
  lab machines.  Please make sure you specify what OS I should use.

\item A directory labeled {\tt src} containing all of your source code
  along with compilation instructions if applicable.

\end{enumerate}

Put all of these items into a folder with the last names of your
group members (no spaces please), zip it up and have one person from your
team submit to blackboard.

\end{document}
